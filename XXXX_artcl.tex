\section{Introduction}

This template provides an overview of typesetting an SRM Article with
\LaTeX. The Source files also provide examples for typical typesetting
techniques.

\section{Filename convention}

All files should be renamed such that the XXXX in the filename becomes
the entry number. If the entry number is XXXX,

\bci
\item Title, abstract, etc. should be stored in  \texttt{XXXX\_header.tex}
\item The article should be stored in \texttt{XXXX\_artcl.tex}
\item Table 1, Table 2 $\ldots$ Table K, should be stored in
  \texttt{XXXX\_tab1.tex}, \texttt{XXXX\_tab2.tex} $\ldots$
  \texttt{XXXX\_tabK.tex} 
\item The Bibliography is created with BibLatex (Biber) from a BibTex
  file in \texttt{XXXX\_bib.bib}
\item Figure 1, Figure 2 $\ldots$ Figure K, should be stored in
  \texttt{XXXX\_fig1.tex}, \texttt{XXXX\_fig2.tex} $\ldots$
  \texttt{XXXX\_figK.tex} (if any)
\item The appendix should be stored in \textrm{XXXX\_appendix.tex} (if
  any) \eci

\section{Typesetting rules}

We follow the rules of the APA Publication Manual. Most of the rules
are implemented thru srm\_main.tex, but please


\bci
\item do not use bold face in the text body
\item do not use vertical lines in tables
\item do not use italics for proper english words in equations should.
\item use identical symbols for math symbols in the text body and in equations
\item take care that separators, hyphen, minus-sign differ in
  length. Look -- as an example -- on on the minus-sign in the equation $1-1=0$
\item prevent parentheses in parentheses in the text body; 
\item use ``double quotation marks''. The rule is: ``We use `single
  quotation marks' only inside  double quotation marks''.
\eci

If in doubt  refer the the APA publication manual, 6th edtion.


\section{Section headings}

We use sections, subsections and subsubsections. Not
more. Never. Unlike APA 6, we we number sections and subsection. The
file \texttt{srm\_main.tex} does this automatically.

\section{Itemize and Enumerate}


Itemlist are typesetted with the ``APAitemize''
environment. Enumeration is typesetted with the ``APAenumerate''
environment. The former can be started/closed with the LaTeX commands
``bci'' and ``eci'', while the latter can be started/closed with
``bce'' and ``ece''. Example. The code

\begin{scriptsize}
\begin{verbatim}
\bce
\item bla bla
\item more bla bla
\ece
\end{verbatim}
\end{scriptsize}

creates this

\bce
\item bla bla
\item more bla bla
\ece

\section{Tables}

Tables are typesetted in a table environment or table* environment --
depending on the size of the table. The table* environment is for wide
tables. Inside the table we allways use threeparttable as shown in the
code below:

\begin{scriptsize}
\begin{verbatim}
\begin{table}
  \begin{threeparttable}[b]
  
  
  
    \caption{The caption of the table}
    \begin{tabular}{l.{2}.{2}}
      \toprule
      Left aligned header & \mc{numheader 1} & \mc{numheader 2} \\ 
      \midrule
      Two digit numbers & 1.34 & 0.20\tmark{a} \\
      More two digit numbers & 1.50 & 1.23 \\ \midrule
      Zero digit number & \mc{300} & \mc{300} \bottomrule
    \end{tabular}
    \vspace{.5em}
    \begin{tablenotes}\small
    \item We start with a general footnote, if any. Note that we don't
      want too many signficance stars.
      
    \item [a] A footnote for footnote signs. Significance footnote
      come last.
      
    \item $*$ p<0.05
    \item $**$ p<0.01
    \item $***$ p<0.001


    \end{tablenotes}
  \end{threeparttable}
\end{table}
\end{verbatim}
\end{scriptsize}

Which leads to Table 1.

\begin{table}
  \begin{threeparttable}[b]
    \caption{The caption of the table}
    \begin{tabular}{l.{2}.{2}}
      \toprule
      Left aligned header & \mc{numheader 1} & \mc{numheader 2} \\ 
      \midrule
      Two digit numbers & 1.34 & 0.20\tnote{a} \\
      More two digit numbers & 1.50 & 1.23 \\ \midrule
      Zero digit number & \mc{300} & \mc{300} \\ \bottomrule
    \end{tabular}
    \vspace{.5em}
    \begin{tablenotes}\small
    \item We start with a general footnote, if any. Note that we don't
      want too many signficance stars. 
      
    \item [a] A footnote for footnote signs. Significance footnote
      come last. 
      
    \item $*$ $p<0.05$
    \item $**$ $p<0.01$
    \item $***$ $p<0.001$
    \end{tablenotes}
  \end{threeparttable}
\end{table}


%%% Local Variables: 
%%% mode: latex
%%% TeX-master: "srm_main"
%%% End: 



\section{Figures}


Figures must be provided as scalable vector graphs, EPS or
PDF. Figures are included in the article as follows:

\begin{verbatim}
\begin{figure}
  \centering
  \includegraphics[width=\linewidth]{XXXX_fig1}
  \caption{The caption as provides by the author}
\end{figure}
\end{verbatim}

\section*{Acknowledgements}

Acknoledgements comes last. They are typsetted with the starred
version of section, i.e. 

\begin{verbatim}
\section*{Acknowledgements}
\end{verbatim}

\section{Bibliography}

We use Biblatex for the Biobliography. You can find the full
descritpion of BibLateX on the internet, but examples for the the main
functions are shown below:

\bci
\item The normal cite is for citations without parentheses. Example:
  \cite[see][pg.\,12]{carrasco03}
\item parencite is for citations in parentheses. Example:
  \parencite[see][pg.\,12]{dept10}
\item textcite if for text citations. Example: 
  \textcite[see][pg.\,12]{dorer11}.
\item For special situations there are also the parencites and
  textcites commands. Here is an example with parencites: 
  \parencites(See)(and the
  introduction)[35]{fitzgerald11}[78]{dorer11}[23]{goerman07}. See the
  BibLatex manual for details. 
\eci

Note that you must use the command ``biber'' to create the actual
Bibliography instead of bibtex. 



%%% Local Variables: 
%%% mode: latex
%%% TeX-master: "srm_main"
%%% End: 
