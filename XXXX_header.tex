%\author{FirstName Surname}
\twoauthors{Georg-Christoph Haas}{Stephanie Eckman}
%\threeauthors{FirstName Surname}{FirstNameSurname}{FirstName Surname}
% etc.

\twoaffiliations{Institute for Employment Research}{RTI International}

\leftheader{Haas and Eckman} 
\rightheader{Linkage Consent \& Measurment Error}

\title{Does Granting Linkage Consent Affect Data Quality?}

\authornote{Contact information:
FirstName Name, Affiliation, Postal addres (email)}

\abstract{Combining survey data with administrative data can make both sources more valuable to researchers, but in most cases respondent consent is needed before such linkage can be done. Research has shown that asking the linkage consent question at the beginning of the questionnaire yields the highest consent rates. However, asking this question at the beginning of the survey may impact respondents' response behavior throughout the rest of the survey. Respondents may feel that their answers are being checked and thus they may make more effort to answer correctly. Alternatively, respondents may feel that their survey responses are not important, because they can be replaced by administrative data and thus they may make less effort to respond correctly. Because many surveys are asking consent questions in the beginning of the survey, it is important to study the effects of granting consent at the beginning of the questionnaire on later response behavior. 

For our analyses we use a survey that asked one half of respondents at the beginning and the other half at the end. We use the entropy balance technique to create a synthetic control group and estimate the effects of the consent question on measurement error and item nonresponse. Our results show that giving consent early in a survey does not affect later responses, which is good news for surveys which seek to maximize their linkage consent rates.
}

\keywords{measurement error; linkage consent; data quality; entropy balance}


%%% Local Variables: 
%%% mode: latex
%%% TeX-master: "srm_main"
%%% End: 

